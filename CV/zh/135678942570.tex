\documentclass[11pt,a4paper,sans]{moderncv}

\usepackage{amsmath}

% moderncv themes
\moderncvstyle{banking}
\definecolor{color0}{rgb}{0,0,0}% black
\definecolor{color1}{rgb}{0.20,0.20,0.95}% red
\definecolor{color2}{rgb}{0.0,0.0,0.0}% dark grey
%\moderncvcolor{red}

%%%%%%%%%%%%%%%%%%%%%%%%%%%%%%%%%%
\renewcommand*{\namefont}{\fontsize{50}{52}\mdseries\upshape}
%%%%%%%%%%%%%%%%%%%%%%%%%%%%%%%%%%

%\usepackage[top=3.2cm, bottom=3.2cm, left=3.2cm, right=3.2cm]{geometry}
\usepackage[scale=0.76]{geometry}
%\usepackage[scale=0.9]{geometry}

\usepackage{xeCJK}
%\setsansfont{Monaco}

\setCJKmainfont{楷体}
\setCJKsansfont{黑体}
\setCJKmonofont{楷体}

% personal data
\firstname{}
\familyname{谭天乐}
%\title{Resumé title (optional)} 
\address{四川省成都市高新区(西区)西源大道2006号}{电子科技大学清水河校区}{邮编:611731}
\mobile{+86~150~0285~0819}                
%\phone{+2~(345)~678~901}                
%\fax{+3~(456)~789~012}                 
\email{a135678942570@gmail.com}
%\extrainfo{additional information}   
%\photo[64pt][0.4pt]{picture}        
%\quote{Some quote (optional)}      

\begin{document}
\makecvtitle
%\maketitle

\section{个人信息}
\cvitem{姓名}{谭天乐}
\cvitem{年龄}{21(1993-09-11)}
\cvitem{毕业时间}{2015-6}
\cvitem{就业意向}{算法工程师,研发工程师}

\section{教育经历}
\cventry{2011--至今}{计算机科学与工程学院 信息安全专业}{\textsf{电子科技大学}}{四川省成都市}
{工学学士}
{}
%{主干课程:
%   离散数学、电路分析基础、模拟电路基础、信号与系统、软件技术基础、数字逻辑设计及应用、电磁场与波、信息论导论、随机信号分析、计算机通信网、
%   微型计算机系统原理及接口应用、数字信号处理、通信原理、移动通信系统。}

% ~\\
% \cventry{2007-2010}{高中}{\textsf{江西省南康中学}}{江西省南康市}
% {}
% {2009年获第十五届全国青少年信息学奥林匹克联赛一等奖保送至电子科技大学。同时还获得哈尔滨工业大学、华南理工大学、大连理工大学的保送生资格。}

%\section{Master thesis}
%\cvitem{title}{\emph{Title}}
%\cvitem{supervisors}{Supervisors}
%\cvitem{description}{Short thesis abstract}

\section{相关经验}

\subsection{竞赛经验}
\cventry{2012--至今}{队员}{\textsf{电子科技大学ACM/ICPC集训队}}{}{}{
    \begin{itemize}
        \item 在ACM/ICPC的竞赛中,需要和另外两名队友合作比赛,积累了大量团队合作解题的经验,有很强的团队意识。
        \item 深入学习研究了各种算法如图论、动态规划、字符串处理、常用数据结构如线段树等,拥有较强的代码能力,能高效的将问题解决并进行编码。
        \item 在TopCoder公司举办的SRM中最高rating为1944 (2014-09-17),在近万名选手中排名前6\%(590名)
        \item 在Codeforces平台上的比赛最高rating为2314 (2014-09-20),在4万余名选手中排名前1\%(230名)
        \item 此类比赛要求在短时间内解决问题并转化为代码,并可以对其他人的代码进行挑战,寻找bug。
    \end{itemize}
}


~~

\cventry{2014}{命题与裁判工作}{\textsf{NOI2014全国青少年信息学奥林匹克竞赛四川代表队选拔赛}}{}{}{
    \begin{itemize}
        \item 国内最高水平信息学竞赛省队队员选拔赛
        \item 要在5小时内解决3道问题,难度极大
        \item 由于四川省内中学间信息学竞赛水平差异较大,命题难度高
    \end{itemize}
}

~~


\cventry{2013~2014}{命题并参与比赛}{\textsf{
2013年及2014年暑期多校联合训练}}{}{}{
    \begin{itemize}
        \item 负责2013年ACM暑期多校训练单场命题工作及2014年多校训练的单场审题与选题工作。
        \item 在2014年ACM暑期多校联合训练中,队伍(team069)总排名第七,高校队伍排名第三。暑期多校联合训练是以10场比赛积分取8场最高分为最终排名,有超过120所高校的近500支队伍参与。
    \end{itemize}
}
\clearpage
\section{荣誉}
\subsection{ACM-ICPC}
\cvline{2014.10.26}{第39届ACM-ICPC亚洲区预选赛(西安赛区) \textsf{~~~~~~~~~~~~~~~~~~~~~~~~~~~~~~~~金奖(季军)}}
\cvline{2014}{第39届ACM-ICPC亚洲区预选赛(西安赛区)邀请赛 \textsf{~~~~~~~~~~~~~~~~~~~~~~~~~~~~~~~~~~~~~~~~金奖}}
\cvline{2014}{第39届ACM-ICPC亚洲区预选赛(上海赛区)邀请赛 \textsf{~~~~~~~~~~~~~~~~~~~~~~~~~~~~~~~~~~~~~~~~金奖}}
\cvline{2014}{2014西南交通大学程序设计竞赛 \textsf{~~~~~~~~~~~~~~~~~~~~~~~~~~~~~~~~~~~~~~~~~~~~~~~~~~~~~~~~~一等奖(亚军)}}
\cvline{2014}{2014四川大学程序设计竞赛 \textsf{~~~~~~~~~~~~~~~~~~~~~~~~~~~~~~~~~~~~~~~~~~~~~~~~~~~~~~~~~~~~~~~一等奖(季军)}}
\cvline{2014}{2014第12届电子科技大学程序设计竞赛 \textsf{~~~~~~~~~~~~~~~~~~~~~~~~~~~~~~~~~~~~~~~~~~~一等奖(季军)}}

~~

\cvline{2013}{第38届ACM-ICPC亚洲区预选赛(长沙赛区) \textsf{~~~~~~~~~~~~~~~~~~~~~~~~~~~~~~~~~~~~~~~~~~~~~~~银奖}}
\cvline{2013}{第38届ACM-ICPC亚洲区预选赛(杭州赛区) \textsf{~~~~~~~~~~~~~~~~~~~~~~~~~~~~~~~~~~~~~~~~~~~~~~~银奖}}
\cvline{2013}{第38届ACM-ICPC亚洲区预选赛(长沙赛区)邀请赛 \textsf{~~~~~~~~~~~~~~~~~~~~~~~~~~~~~~~~~~~~~~金奖}}
\cvline{2013}{2013第11届电子科技大学程序设计竞赛 \textsf{~~~~~~~~~~~~~~~~~~~~~~~~~~~~~~~~~~~~~~~~~~~~~~~~~~~~~~~~~~一等奖}}

~~

\cvline{2012}{第37届ACM-ICPC亚洲区预选赛(金华赛区) \textsf{~~~~~~~~~~~~~~~~~~~~~~~~~~~~~~~~~~~~~~~~~~~~~~~铜奖}}
\cvline{2012}{2012四川大学程序设计竞赛 \textsf{~~~~~~~~~~~~~~~~~~~~~~~~~~~~~~~~~~~~~~~~~~~~~~~~~~~~~~~~~~~~~~~~~~~~~一等奖}}
\cvline{2012}{2012第10届电子科技大学程序设计竞赛 \textsf{~~~~~~~~~~~~~~~~~~~~~~~~~~~~~~~~~~~~~~~~~~~~~~~~~~~~~~~~~二等奖}}

\section{专业技能}
\subsection{语言技能}
\cvitem{\textsf{全国大学英语四级考试}}{通过}
\cvitem{\textsf{全国大学英语六级考试}}{通过}
\cvitem{\textsf{第二届CCF软件能力认证考试}}{370分(6.46\%)}

\subsection{计算机技能}
\cvitem{\textsf{版本控制}}{Git}
\cvitem{\textsf{编程语言}}{C, C++, Java, Python}
\cvitem{\textsf{其它工具}}{\LaTeX}

\subsection{其它技能}
\cvitem{\textsf{问题解决}}{能在较短时间内独立思考、周密分析问题并形成解决问题的思路;由于经常参加程序竞赛,提高了数学英语功底、提升了算法设计能力和编程技巧、养成了良好的协作精神、锻炼了心理素质和临场应变能力}
\cvitem{\textsf{学习能力}}{对新技术有浓厚的兴趣,有很强的自学能力}
\cvitem{\textsf{团队合作}}{拥有良好的沟通能力,有很强的团队意识}

\end{document}
